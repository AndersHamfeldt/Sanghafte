\begin{song}{Så lunka vi}{fredmanssangno21}
\kom{Fredmans sång n:o 21}
\begin{vers}
Så lunka vi så småningom \\
Från Bacchi buller och tumult\\
När döden ropar - Granne kom\\
Ditt timglas är nu fullt\\
Du gubbe fäll din krycka ner\\
Och du du yngling, lyd min lag\\
Den skönsta nymf som mot dig ler\\
Inunder armen tag\\
Tycker du att graven är för djup\\
Nå välan, så tag dig då en sup\\
Tag dig sen dito en, dito två, dito tre\\
Så dör du nöjdare\\
\end{vers}
\begin{vers}
Men du som med en trumpen min, \\
Bland riglar, galler, järn och lås\\
Dig vilar på ditt penningskrin\\
Inom din stängda bås\\
Och du som svartsjuk slår i kras\\
Buteljer, speglar och pokal;\\
Bjud nu god natt, drick ut ditt glas\\
Och hälsa din rival\\
Tycker du ...\\
\end{vers}
\newp
\begin{vers}
Men du som med en ärlig min\\
Plär dina vänner häda jämt\\
Och dem förtalar vid ditt vin\\
Och det liksom på skämt\\
Och du som ej försvarar dem\\
Fastän ur deras flaskor du\\
Du väl kan slicka din fem\\
Vad svarar du väl nu?\\
-Tycker du att..\\
\end{vers}
\begin{vers}
Säg är du nöjd, min granne säg, \\
Så prisa värden nu till slut\\
Om vi ha en och samma väg\\
Så följoms åt - drick ut\\
Men först med vinet rött och vitt\\
För vår värdinna bugom oss\\
Och halkom sen i graven fritt\\
Vid aftonstjärnans bloss\\
Tycker du ..\\
\end{vers}
\end{song}
